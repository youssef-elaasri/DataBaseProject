\chapter{Bilan du projet}
\section{Organisation}
\subsection{Analyse Approfondie et Collaboration Client}
Notre démarche a débuté par une lecture attentive et approfondie du texte de description du projet, paragraphe par paragraphe. L'objectif premier était de bien appréhender les besoins du client. Pour ce faire, des échanges fréquents avec le client ont été instaurés afin de garantir que les étapes suivantes du projet répondraient de manière adéquate à ses besoins. En parallèle de ces échanges, une analyse approfondie du problème a été entreprise pour extraire les dépendances fonctionnelles et les contraintes. Il était également crucial de anticiper les étapes futures du projet pour assurer une cohérence globale. Cette phase a permis à chaque membre de l'équipe de se projeter dans le déroulement du projet.
\subsection{Conception du Diagramme Entités/Associations (UML)}
Après la conclusion du premier sprint d'analyse, nous avons entamé la conception du Diagramme Entités/Associations (UML). Cette étape a marqué une division des tâches au sein de l'équipe, avec une répartition des différentes tables à construire. Étant donné que ces tables étaient interconnectées, le travail s'effectuait simultanément, nécessitant une communication constante entre les membres pour garantir la cohérence globale du projet. Des appels réguliers à un expert en bases de données ont été réalisés pour valider la pertinence de nos choix et prendre des décisions majeures.
\subsection{Implémentation et Développement}
Le dernier sprint du projet a impliqué la construction du script SQL pour générer les tables et les remplir, ainsi que la création des fonctions JAVA pour interagir avec la base de données. Grâce à la vision commune partagée par tous les membres de l'équipe, nous avons pu diviser ces phases de développement de manière efficace. Cependant, des échanges réguliers ont été maintenus pour s'assurer de la cohérence du travail accompli. Une décision stratégique a été prise pour centraliser l'ensemble du projet sur un dépôt Git, permettant un accès fluide au travail des autres membres à tout moment. Cette approche collaborative a renforcé la coordination au sein de l'équipe et a favorisé une progression harmonieuse du projet.



\section{Gestion des Points Difficiles}

La gestion des dépendances fonctionnelles complexes entre les entités du projet s'est avérée être un défi majeur. Pour surmonter cela, nous avons opté pour une approche collaborative renforcée par l'utilisation de Git en tant que plateforme centrale. Cela a permis à chaque membre de l'équipe d'accéder en temps réel aux contributions des autres, facilitant ainsi la synchronisation des changements et assurant une vue d'ensemble cohérente du projet.

Lors de la conception du Diagramme Entités/Associations (UML), la complexité des discussions autour des choix de modélisation a été atténuée en utilisant draw.io, une plateforme collaborative de création de diagrammes. Cela a permis à chaque membre de contribuer simultanément à la conception, de partager des idées visuelles, et d'ajuster le diagramme en temps réel. Cette approche a considérablement amélioré la compréhension collective et la clarté des décisions prises.

La communication avec le client s'est parfois révélée être un défi, mais nous avons adapté notre approche en formulant des questions courtes et précises. Cela a permis d'obtenir des réponses rapides et spécifiques, facilitant ainsi la prise de décision et l'avancement du projet malgré les contraintes de communication.

En résumé, la centralisation des travaux sur Git, l'utilisation d'outils collaboratifs tels que draw.io, des questions précises lors des échanges avec le client, et une collaboration étroite pour résoudre les problèmes techniques ont été les clés du succès pour surmonter les points difficiles rencontrés durant le projet.