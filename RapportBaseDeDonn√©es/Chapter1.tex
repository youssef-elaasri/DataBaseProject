\chapter{Documentation du projet}
\section{Analyse du Problème}

Le projet débute par l'analyse du texte expliquant les besoins du client. Notre stratégie consistait à identifier les objets nécessaires pour construire les tables de notre future base de données, en incluant les dépendances fonctionnelles entre eux ainsi que les contraintes de valeurs et de multiplicité.

L'email du refuge a été utilisé pour déterminer tous les attributs de la table \texttt{"refuge"}. De plus, en connaissant l'objet \texttt{"Repas"}, nous avons pu établir une relation permettant de déduire le prix associé à ce repas. L'idée d'avoir une propriété propre a commencé à prendre forme.

Il a été décidé qu'un refuge peut disposer d'au plus un numéro. En ce qui concerne les \texttt{"Formations"}, nous avons observé qu'une formation peut proposer au moins une activité, introduisant ainsi une nouvelle contrainte de multiplicité.

Pour assurer la conformité de notre base de données au Règlement Général sur la Protection des Données (RGPD), nous avons créé un identifiant utilisateur (\texttt{"IdUsr"}) pour préserver les traces de réservations sans avoir besoin de stocker les données clients directement. Ainsi, les données clients sont mieux protégées et peuvent être supprimées si le client en fait la demande.

Afin de bien identifier les réservations pour les refuges et les formations, des identifiants ont été introduits pour jouer le rôle de clés primaires.

















\subsection{Conception du Diagramme Entités/Associations (UML)}

La deuxième étape a été la conception du diagramme Entités/Associations (UML). La table \texttt{"Refuge"} regroupe toutes les informations relatives au refuge. Afin de respecter la contrainte de multiplicité stipulant qu'un refuge peut avoir au plus un numéro, nous avons établi une association sémantique entre la table \texttt{"Refuge"} et la table \texttt{"Ref\_NumTel"}. Une propriété propre relie également \texttt{"Refuge"} à \texttt{"Repas"}, permettant d'indiquer le prix d'un repas dans un refuge spécifique.

La table \texttt{"ReservationRefuge"} est naturellement liée à la table \texttt{"Refuge"} pour indiquer à quel refuge la réservation correspond. Elle est également liée à la table \texttt{"CompteUtilisateur"}, permettant d'identifier à qui appartient cette réservation.

Les données utilisateur sont stockées dans une table distincte du même nom, associée à la table \texttt{"CompteUtilisateur"}, qui ne contient que l'identifiant de l'utilisateur. Cela garantit la conformité au RGPD. De plus, étant donné qu'un utilisateur peut être adhérent ou non, la table \texttt{"Adherents"} est une sous-entité faible de la table \texttt{"Utilisateur"}.

Cette dernière est associée aux tables \texttt{"ReservationFormation"} et \texttt{"LocationMateriel"}. Ces deux tables sont respectivement liées à la table \texttt{"Formations"} par une association sémantique, et à la table \texttt{"LotMateriel"} par une propriété propre déterminant le nombre de pièces réservées et le nombre cassées/perdues.

La table \texttt{"LotMateriel"} est associée à une table texte pour indiquer si un lot peut ou non avoir une description. Elle est également associée à une seule et unique catégorie dans la table \texttt{"Categorie"}. Cette dernière possède une association réflexive permettant de représenter le fait qu'une catégorie mère peut avoir plusieurs sous-catégories ou aucune.

La table \texttt{"Activité"} est associée à \texttt{"Formation"} pour indiquer de quelle formation il s'agit. Elle est également associée à la table \texttt{"LotMateriel"}, indiquant le matériel utilisé pour une certaine activité.


\section{Passage au Relationnel}

Le modèle relationnel de la base de données reflète la structure définie dans le diagramme Entités/Associations (UML) et se traduit en un ensemble de tables interconnectées. Chaque table représente une entité distincte avec ses attributs, les clés primaires et les clés étrangères nécessaires pour maintenir l'intégrité des données.

\begin{enumerate}
     


\item    CompteUtilisateur : La table \textbf{CompteUtilisateur} contient les informations relatives aux utilisateurs, avec un identifiant unique (\texttt{idUsr}). L'entité \textbf{Utilisateur} est associée à cette table, utilisant l'identifiant \texttt{idUsr} comme clé étrangère.

\item    Adherent : L'entité \textbf{Adherent} est représentée par une table du même nom, reliant les utilisateurs adhérents via l'identifiant \texttt{idUsr}, qui agit comme clé primaire et clé étrangère vers \textbf{CompteUtilisateur}.

 \item   Refuge : Les informations sur les refuges sont stockées dans la table \textbf{Refuge}, avec l'email du refuge comme clé primaire. La table \textbf{Propose} associe les repas proposés par chaque refuge avec leurs prix spécifiques.

\item    ReservationRefuge : La réservation d'un refuge par un utilisateur est enregistrée dans la table \textbf{ReservationRefuge}, avec un identifiant unique (\texttt{idResRefuge}). Les liens vers l'utilisateur adhérent et le refuge sont établis via les clés étrangères.

\item    Formation : Les détails des formations sont capturés dans la table \textbf{Formation}, avec l'année et le rang comme clés primaires. La table \textbf{ReservationFormation} enregistre les réservations des formations par les adhérents.

\item    Utilise : L'utilisation du matériel par une activité est modélisée par la table \textbf{Utilise}, reliant les activités aux lots de matériel spécifiques.

\item    LotMateriel : Les caractéristiques des lots de matériel sont enregistrées dans la table \textbf{LotMateriel}, avec une clé primaire composée de la marque, du modèle, et de l'année. La présence d'une date de péremption est facultative.

\item    LocationMateriel : Les locations de matériel effectuées par les adhérents sont enregistrées dans la table \textbf{LocationMateriel}, avec un identifiant unique (\texttt{idLocationMateriel}) et des références aux adhérents.

\item    ReservationPieces : Les réservations spécifiques de pièces de matériel sont gérées par la table \textbf{ReservationPieces}, reliant les réservations aux lots de matériel et aux locations.

\item    DatePeremption : La table \textbf{DatePeremption} stocke les dates de péremption associées aux lots de matériel, liées par la table \textbf{A\_pour\_datePeremption}.
\end{enumerate}
Chaque table est conçue pour refléter les relations et les contraintes établies lors de la phase de conception, garantissant ainsi une représentation cohérente des données du projet.

\subsection{Formes Normales}

\section{Analyse des Fonctionnalités}

\subsection{Parcours des formations/refuges/matériels disponibles}
\subsection{Réservation de Refuge}
\subsection{Réservation de Formation}


\subsection{Location et Retour de Matériel}
La location et le retour du matériel sont des étapes clés du cahier des charges, et nous nous sommes efforcés de respecter les souhaits du client.

Pour la location de matériel, l'utilisateur est requis de nous fournir le ou les lots qu'il souhaite réserver. En effet, l'utilisateur a la possibilité de louer plusieurs lots au sein d'une même réservation.

 Chaque réservation est identifiée par un numéro unique, idLocationMateriel, propriété que nous avons ajoutée dans la table LocationMateriel pour faciliter la traçabilité de chaque transaction, y compris le retour du matériel.


Chaque lot de matériel est identifié par la marque, le modèle et l'année.  L'utilisateur nous transmet ces informations, ainsi que le nombre d'unités qu'il souhaite louer, accompagné des dates de récupération et de retour de location.  

Une fois ces détails reçus, nous procédons à plusieurs vérifications pour garantir que la location puisse être effectuée. Ces vérifications englobent divers critères.

\begin{enumerate}
    \item La vérification de la disponibilité du matériel\\
    Pour évaluer la disponibilité des lots demandés par l'utilisateur, nous examinons les réservations existantes ainsi que le nombre total d'unités disponibles. En cas d'insuffisance d'unités, un message est immédiatement affiché à l'utilisateur, l'informant de la contrainte.
    Ceci grâce à la requête sql suivante (Les "?" remplaçant les informations fournies): \\
    \\
        SELECT lm.modele,lm.marque,lm.annee,(lm.nbPieces - rp.nbPiecesReservees - rp.nbPiecesCasseesPerdues)
                FROM ReservationPieces rp \\
                JOIN  LotMateriel lm \\
                ON rp.marque = lm.marque and rp.modele = lm.modele and rp.annee = lm.annee \\
                JOIN LocationMateriel locm \\
                ON rp.idLocationMateriel = locm.idLocationMateriel\\ 
                WHERE locm.dateRetour >= ? and locm.dateRecup <= ? \\
                UNION \\
                SELECT lm.modele,lm.marque,lm.annee,lm.nbPieces \\
                FROM LotMateriel lm \\
                WHERE (lm.modele, lm.marque, lm.annee) NOT IN \\
                ( SELECT rp.modele, rp.marque, rp.annee FROM ReservationPieces rp)\\
    La première section de cette requête examine la disponibilité en fonction des locations déjà effectuées. La deuxième section vise à s'assurer que l'utilisateur ne demande pas plus d'unités que celles disponibles en stock initial.
    \item La vérification de la durée de la location\\
    Tout d'abord, nous veillons à ce que l'utilisateur fournisse des dates valides, impliquant une date de récupération postérieure à la date actuelle et une date de retour postérieure à la date de récupération. Cette étape garantit l'intégrité chronologique du processus de location.
    
    En parallèle, nous effectuons une vérification pour nous assurer que la durée de la location n'excède pas deux semaines. Cette limitation vise à répondre aux paramètres définis pour une gestion optimale des réservations. En cas de non-conformité à ces critères, des messages d'erreur sont générés pour guider l'utilisateur dans la soumission de dates adéquates.
\end{enumerate}
Une fois que toutes les vérifications ont été effectuées et que tout est conforme, la location est confirmée. Au moyen d'une requête sql "INSERT INTO", les informations correspondantes sont ajoutées à la table LocationMateriel, ainsi qu'à la table ReservationsPieces pour assurer une gestion complète des réservations, y compris les détails spécifiques des pièces réservées.

En ce qui concerne le retour de matériel, nous partons du principe que l'utilisateur effectue le retour dans les délais convenus. Il doit nous fournir les détails du lot qu'il souhaite retourner, ainsi que le nombre éventuel de pièces abîmées/perdues. En cas de pièces cassées, le cahier des charges nous oblige à les retirer du lot. Toutefois, pour assurer la traçabilité, ces pièces sont conservées dans le total du lot, c'est-à-dire dans la table LotMateriel, mais elles sont exclues du nombre de pièces disponibles pour les futures réservations.

Par ailleurs, nous effectuons le calcul du coût des pièces endommagées, que nous ajoutons à la somme due par l'utilisateur dans la table Utilisateur. Cette approche garantit une gestion précise des retours, en tenant compte des dommages éventuels et de leur impact sur la facturation globale de l'utilisateur.







