
\DeclareNameAlias{sortname}{last-first}
\DeclareFieldFormat{edition}{%
  \ifinteger{#1}
    {\ifnumequal{#1}{1}%
     {}%
     {\mkbibordedition{#1}~\bibstring{edition}}%
    }
    {#1\isdot}}

\DeclareFieldFormat[article,inbook,incollection]{title}{#1\isdot}
\DeclareFieldFormat[article,inbook,incollection]{citetitle}{#1\isdot}

\newrobustcmd{\MakeTitleCase}[1]{%
  \ifboolexpr{test {\ifentrytype{article}} or test {\ifentrytype{inbook}} or test {\ifentrytype{incollection}}}
    {#1}
    {\MakeSentenceCase{#1}}}

\DeclareFieldFormat{urldate}{\bibsentence\mkbibbrackets{\bibstring{urlseen}\space#1}}
\DeclareFieldFormat{url}{\bibstring{urlfrom}\addcolon\space\url{#1}}

\renewbibmacro*{journal}{%
  \iffieldundef{journaltitle}
    {}
    {\printtext[journaltitle]{%
       \printfield[titlecase]{journaltitle}%
       \setunit{\subtitlepunct}%
       \printfield[titlecase]{journalsubtitle}}
       \ifboolexpr{
         not test {\iffieldundef{url}}
         or
         not test {\iffieldundef{urldate}}
         or
         not test {\iffieldundef{doi}}
         or
         not test {\iffieldundef{eprint}}
       }
         {\nopunct\bibstring[\mkbibbrackets]{online}}%
         {}}}

\newbibmacro*{journal+issuetitle}{%
  \usebibmacro{journal}%
  \setunit*{\addspace}%
  \iffieldundef{series}
    {}
    {\newunit
     \printfield{series}%
     \setunit{\addspace}}%
  \newunit
  \usebibmacro{volume+number+eid}%
  \setunit{\addspace}%
  \usebibmacro{issue+date}%
  \setunit{\addcolon\space}%
  \usebibmacro{issue}%
  \newunit}

\NewBibliographyString{online}
\DefineBibliographyStrings{french}{%
  urlseen    = {accessed},
  online     = {online},
}
%\addbibresource{example.bib}
\renewcommand*{\nameyeardelim}{\addcomma\addspace}
\renewbibmacro{in:}{%
  \ifentrytype{article}{}{\printtext{\bibstring{in}\intitlepunct}}} % removes "In" preceeding journal title

% Define SQL language settings
\lstdefinestyle{SQL}{
  language=SQL,
  basicstyle=\ttfamily\small,
  commentstyle=\color{gray},
  keywordstyle=\color{blue},
  numbers=left,
  numberstyle=\tiny,
  numbersep=5pt,
  frame=single,
  breaklines=true,
  breakatwhitespace=true,
  tabsize=2
}

\usepackage{float}